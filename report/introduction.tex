% !TEX root = ./final_report.tex
\section{Introduction \label{sec:intro}}
	Unmanned Aerial Vehicles have seen an acceleration in interest for solutions to problems within environments that are cluttered with obstacles.  In most cases not all obstacles are known at the initial path-planning stage, or that accurate representation of all known obstacles in memory is not feasible.  This constraint gives rise to the need for an online path-planning methodology where the vehicle can sense its environment in and modify the flight path in real time.  

	For this project we considered the application of autonomous package delivery within an urban environment containing obstacles that are unknown before the mission begins.  The vehicle begins at a specified start location, and must navigate the environment to reach the target delivery location without colliding with an obstacle.  The overall strategy implemented to achieve this follows.  At a specified frequency, the control loop first senses the environment to look for any new obstacles.  If an obstacle is found that intersects the current planned path, a new path is computed.  Once a path has been chosen, the control inputs to optimally track the path towards the goal are found.  The inputs are then applied to the control until the the next period begins.